\documentclass[10pt,a4paper,twocolumn]{article}

\usepackage[utf8]{inputenc}
\usepackage[spanish]{babel}
\usepackage{listings}
\usepackage{geometry}
\usepackage{color}

\definecolor{dkgreen}{rgb}{0,0.6,0}
\definecolor{gray}{rgb}{0.5,0.5,0.5}
\definecolor{mauve}{rgb}{0.58,0,0.82}

\geometry{verbose,landscape,letterpaper,tmargin=1cm,bmargin=1.5cm,lmargin=1cm,rmargin=1cm}

\setlength{\columnsep}{0.15in}
\setlength{\columnseprule}{1px}

\lstset{frame=tb,
  language=C++,
  aboveskip=3mm,
  belowskip=3mm,
  showstringspaces=false,
  columns=flexible,
  basicstyle={\small\ttfamily},
  numbers=none,
  numberstyle=\tiny\color{gray},
  keywordstyle=\color{blue},
  commentstyle=\color{dkgreen},
  stringstyle=\color{mauve},
  breaklines=true,
  breakatwhitespace=true,
  tabsize=3
}

\begin{document}
	\title{Manual de CP}
	\author{ErrorByNight}
	\date{}
	\maketitle
	
	\tableofcontents
	\newpage
	\section{Plantilla}
		\lstinputlisting{./src/template.cpp}
	\newpage
	
	\section{Teoria de numeros}
	\subsection{Criba de Eratóstenes}
	\lstinputlisting{./src/teoria_numeros/eratostenes.cpp}
	
	\subsection{Big Mod}
	\lstinputlisting{./src/teoria_numeros/bigmod.cpp}
	
	
	\section{Grafos}
	\subsection{DFS}
		Complejidad: O(n+m) donde n es el numero de nodos y m es el numero de aristas
		\lstinputlisting{./src/Grafos/dfs.cpp}
		
		\subsection{BFS}
Complejidad: O(n+m) donde n es el numero de nodos y m es el numero de aristas
		\lstinputlisting{./src/Grafos/bfs.cpp}
		
		\subsection{Topological Sort}
Complejidad: O(n+m) donde n es el numero de nodos y m es el numero de aristas
		\lstinputlisting{./src/Grafos/topo_sort.cpp}
		 
		
		\subsection{Dijkstra Algorithm}
Complejidad: $O(n^2+m)$ donde n es el numero de nodos y m es el numero de aristas
		\lstinputlisting{./src/Grafos/dijkstra.cpp}
		
		
		\subsection{Kruskal}
Complejidad: $O(nlog(m))$ donde n es el numero de nodos y m es el numero de aristas
		\lstinputlisting{./src/Grafos/Kruskal.cpp}
		
		
		\subsection{Puntos de articulacion}
Complejidad: $O(nlog(m))$ donde n es el numero de nodos y m es el numero de aristas
		\lstinputlisting{./src/Grafos/puente.cpp}
		
		
		\section{Estructuras}
		\subsection{Trie Tree}
		\lstinputlisting{./src/estructuras/trie_tree.cpp}
		
		\subsection{Segment Tree}
		\subsubsection{Iterativo}
		\lstinputlisting{./src/estructuras/segment_iterativo.cpp}
		
		\subsubsection{Recursivo}
		\lstinputlisting{./src/estructuras/segment_recursivo.cpp}
		
		\subsubsection{Persistente}
		\lstinputlisting{./src/estructuras/segment_tree_persistente.cpp}
		
\end{document}